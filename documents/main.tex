\documentclass [a4paper,11pt,fleqn]{report}

\usepackage {fancyhdr}
\pagestyle {fancy}

\usepackage {amsmath, amsfonts}

\title {JavaScript Debugger\\Using Data Structure Visualization\\JavaScriptのデータ構造可視化\\を用いたデバッガ}
\author {東京工業大学大学院\\情報理工学研究科数理・計算科学\\学籍番号:12M54060\\徐駿剣}
\date {\today}

\begin {document}

\maketitle

\begin {abstract}
Debugger is used to test and debug target programs by stepping through the program and examining the current program state by evaluating the values of expressions or checking the stack trace. However, the character-based expressions evaluation provides limited insight into the current program state. This paper examines how a debugger can provide a higher-level, more informative visualization based on data structure. Except static view, the debugger also allows generating animation while target object is modified. However, high-level visualization typically relies on user augmented source code. This paper enables the debugger to understand the data structure of target object by externally supplied semantic information which is written in a declarative language.
\end {abstract}

\tableofcontents
\listoffigures
\listoftables

\chapter {Introduction}

\section {Motivation}
Program visualization systems translate program into visual shapes. It is often used in algorithm animation systems where algorithm behavior is visualized by producing the abstraction of the data and the operations of the algorithm. Such visualization also can be used in a debugger, as it allows for better understanding on the behavior of the program and help perceive which parts of the program does not function correctly.

A debugger is used to test and debug target programs. It offers many sophisticated functions such as running a program step by step, pausing the program at some event or specified instruction by means of a breakpoint, and examining the current state by evaluating the values of expressions or checking the stack trace. However, the character-based expressions evaluation provides limited insight into the current program state. Modern IDEs intend to solve this problem by offering a dedicated view to watch expressions. As a typical example of object-oriented programming language, when the variable being watched is of reference type, its fields is listed in a tree-view table. The root of this list represents the value itself. If a field is also of reference type, it can be further expanded in the same manner. Fields of primitive type can not be further expanded.

However, such visualization still has deficiencies. Firstly, all of the objects are visualized in a tree-view table, and hence lack of important semantic information. Secondly, The whole view will be refreshed while stepping through the program. It is nearly impossible to check the modification parts and find the relevance between two succeeding states.

\section {Main Contribution}
This paper examines how a debugger can visualize data structure and generate smooth animation automatically while according object is modified. The user has to write VisualJS, a new declarative language describing semantic information of the data structures, by himself. Instead of displaying object in a general way, we wish to produce more helpful and informative visualization.

\section {Outline of the Thesis}
The structure of the thesis is as follows. Chapter \ref{Related Research} gives an introduction to the visual debugging and program visualization as related research. We review a series of typical systems and compare them with our research. In Chapter \ref{System Proposal}, we introduce the design of a new declarative language called VisualJS, and explain how to make use of VisualJS to establish the mapping relationship between a object graph and visual representations. In Chapter \ref{System Implementation}, we discuss the implementation details of the system. In Chapter \ref{Examples}, we evaluate the system via several examples. Finally, Chapter \ref{Summary} concludes the thesis including a discussion of possible future work.

\bibliographystyle {plain}
\bibliography {/Users/junjianxu/Dropbox/projects/visualjs/documents/reference}

\end {document}
