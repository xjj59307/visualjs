\documentclass [a4paper,11pt,fleqn]{report}

\usepackage {listings}
\lstdefinelanguage {javascript}{
  keywords={typeof, new, true, false, catch, function, return, null, catch, switch, var, if, in, while, do, else, case, break},
  ndkeywords={class, export, boolean, throw, implements, import, this},
  sensitive=false,
  comment=[l]{//},
  morecomment=[s]{/*}{*/},
  morestring=[b]',
  morestring=[b]"
}
\lstset {
  language=javascript,
  extendedchars=true,
  basicstyle=\footnotesize\ttfamily,
  showstringspaces=false,
  showspaces=false,
  tabsize=2,
  breaklines=true,
  showtabs=false,
  captionpos=b
}

\usepackage {parskip}

\usepackage[toc]{appendix}

\usepackage [nounderscore]{syntax}

\shortverb{\|}
\setlength{\grammarindent}{2cm}
\newcommand{\indalt}[1][2]{\\\hspace*{#1em}\textbar\quad}

\usepackage [dvipdfmx,hiresbb] {graphicx}
\usepackage [dvipdfmx] {color}
\usepackage {url}

\usepackage {fancyhdr}
\fancyhf{}
\fancyfoot[CO,CE]{\thepage}
\fancyhead[LO]{\leftmark}
\fancyhead[RE]{\rightmark}
\pagestyle {fancy}

\usepackage {amsmath, amsfonts}
\usepackage {setspace}
\usepackage {titlesec}

\title {JavaScript Debugger\\Using Data Structure Visualization\\(JavaScriptのデータ構造可視化\\を用いたデバッガ)}
\author {東京工業大学大学院\\情報理工学研究科数理・計算科学\\学籍番号:12M54060\\徐駿剣\\\\2013年度修士論文\\指導教員 脇田 建 准教授}
\date {\today}

\begin {document}

\maketitle

\begin {abstract}
Debugger is used to test and debug target programs by stepping through the program and examining the current program state by evaluating the values of expressions or checking the stack trace. However, the character-based expressions evaluation provides limited insight into the current program state. This thesis examines how a debugger can provide a higher-level, more informative visualization based on data structure. Except static view, the debugger also allows generating animation while target object is modified. However, high-level visualization typically relies on user augmented source code. This thesis enables the debugger to understand the data structure of target object by externally supplied semantic information which is written in a declarative language called visualjs.
\end {abstract}

\tableofcontents
\listoffigures

\chapter {Introduction}

\section {Motivation}
Program visualization systems translate program into visual shapes. It is often used in algorithm animation systems where algorithm behavior is visualized by producing the abstraction of the data and the operations of the algorithm. Such visualization also can be used in a debugger, as it allows for better understanding on the behavior of the program and help perceive which parts of the program does not function correctly.

A debugger is used to test and debug target programs. It offers many sophisticated functions such as running a program step by step, pausing the program at some event or specified instruction by means of a breakpoint, and examining the current state by evaluating the values of expressions or checking the stack trace. However, the character-based expressions evaluation provides limited insight into the current program state. Modern IDEs intend to solve this problem by offering a dedicated view to watch expressions. As a typical example of object-oriented programming language, when the variable being watched is of reference type, its fields is listed in a tree-view table. The root of this list represents the value itself. If a field is also of reference type, it can be further expanded in the same manner. Fields of primitive type can not be further expanded.

However, such visualization still has deficiencies. Firstly, all of the objects are visualized in a tree-view table, and hence lack of important semantic information. Secondly, The whole view will be refreshed while stepping through the program. It is nearly impossible to check the modification parts and find the relevance between two succeeding states.

\section {Main Contribution}
This paper examines how a debugger can visualize data structure and generate smooth animation automatically while according object is modified. The user has to write VisualJS, a new declarative language describing semantic information of the data structures, by himself. Instead of displaying object in a general way, we wish to produce more helpful and informative visualization.

\section {Outline of the Thesis}
The structure of the thesis is as follows. Chapter \ref{Related Research} gives an introduction to the visual debugging and program visualization as related research. We review a series of typical systems and compare them with our research. In Chapter \ref{System Proposal}, we introduce the design of a new declarative language called VisualJS, and explain how to make use of VisualJS to establish the mapping relationship between a object graph and visual representations. In Chapter \ref{System Implementation}, we discuss the implementation details of the system. In Chapter \ref{Examples}, we evaluate the system via several examples. Finally, Chapter \ref{Summary} concludes the thesis including a discussion of possible future work.
\chapter {Related Research}

Price et al. \cite{Price1993211} have defined software visualization as the use of the crafts of typography, graphic design, animation, and cinematography with modern human computer interaction technology to facilitate both the human understanding and effective use of computer software. This is a very abstract definition. 

\section {Debugger}

\section {Visual Debugger}
\chapter {System Proposal}
To solve the problems mentioned in last chapter, our system used traversal-based method proposed by \cite{729554} to traverse object to design a new declarative language oriented for JavaScript. On top of this, we proposed a mapping mechanism to generate the mapping relationship from object graph to visual nodes. Making use of the mapping relationship, the system is able to generate visualization and animate it automatically while stepping through program. Instead of providing rich kinds of data structure visualization, our research focuses on how users can display and animate the objects they are interested in handily and how the system works in the background.

\section {From Object to Visual Shapes}
To generate valid data structure visualization and animation, we must at first prove the consistency between the program and visualization. A program is composed of a series of instructions which are executed orderly. Under the control of debugger, program is being paused until step requests come. Every time the program pauses, it represents a new program state. 

What need to be proved is that the initial program state is being correctly visualized and every time the new program state is generated, the animation will respond to it correctly. However, program state contains too much information. A computer program stores data in variables, which represent storage locations in the computer's memory. Program state contains all contents of these memory locations. In contrast, users always have limited interest and perception at a time. We intend to help users understand the program from any angel he is interested in. Instead of visualizing the whole program state like Heapviz \cite{Aftandilian:2010:HIH:1879211.1879222}, this system always focuses on visualizing one object but allows switching targets at any time. 

We have confirmed that the target to be visualized is an object. This seems trivially different from previous research like prestigious data structure visualization system, jGRASP \cite{Cross:2007:DOV:1227310.1227316} that always uses variable name as the target of visualization. Although there is no problem using variable name previously because they are all about static visualization. Our research introduces animation hence the problem has tremendously changed. The fundamental difference is whether there exists substantive relationship between two consecutive states. Animation is visualizing the changes between two consecutive states, so it have to proceed on the former state. That is why variable name can not be used here as the target of visualization in that even the variable with the same name may refer to a different object or even an empty object in different program state. Both situations may lead to meaningless visualization because original mapping relationship can not be adapted for new object with different structure.

\subsection {Mapping Mechanism}
Object graph details the relationships between objects. We can get one object graph \(G = (V, E)\) for a given object at certain program state. The vertex set $V_p$ is defined by objects which refer to that given object directly or indirectly, and the edge set $E$ is defined by reference relationships among those objects.

By iterating the object graph using next actions, we will get a subgraph of it, \(G_p = (V_p, E_p)\). Visual nodes are defined based on this graph. 

Visual nodes including two types, node and edge, are created by create actions. They are also a graph, \(G_v = (V_v, E_v)\). The vertex set is defined by visual nodes of node type. They are related to $V_p$. The relationships can be defined as a non-injective and surjective function \(f:V_p\rightarrow V_v\) (\textbf{\textit{TODO}}: illustration). The edge set is defined by visual nodes of edge type.

\subsection {Animation Semantics}

\section {Declarative language}
The declarative language is used to generate visual 

The principles that were considered when we designed this system are as follows:
\begin {enumerate}
\item \textbf {Expression Ability}
\item \textbf {Flexibility}
\end {enumerate}

\subsection {Data Model}
Although this system is constructed based on JavaScript, it theoretically suits all object-oriented programming languages like C++ and Java and any other programming language in which an object is constructed in a recursive way, which means an object is composed of other primitive type values or objects. Hence the object can be traversed and matching actions can be executed to generate the topology of visual nodes.

\subsection {Patterns}
A pattern aligns a series of predicates to be matched when object comes. The first action whose predicate is matched will be executed.

\subsection {Actions}

\section {Data Structure Visualization and Animation}

\chapter {System Implementation}

\section {V8 Debugger Protocol}
V8 is able to debug the JavaScript code running in it. The debugger related API can be used in two ways, a function based API using JavaScript objects and a message based API using JSON based protocol. The function based API is for in-process usage, while the message based API is for out-process usage. This system is implemented with message based API. The protocol packet is defined in JSON format and can to be converted to string.

\subsection {Protocol Packet Format}
All packets have two basic elements called seq and type. The seq field holds the consecutive assigned sequence number of the packet. And type field is a string value representing the packet is request, response or event. Each request will receive a response with the same request seq number as long as the connection still works. And additional events will be generated on account of particular requests or system errors. Each packet has the following structure.

\begin{lstlisting}[language=JavaScript]
{
	"seq" : <number>,
	"type": <type>,
	...
}
\end{lstlisting}

A request packet has the following structure.

\begin{lstlisting}[language=JavaScript]
{
	"seq"      : <number>,
	"type"     : "request",
	"command"  : <command>,
	"arguments": { ... }
}
\end{lstlisting}

A response packet has the following structure. If command fails, the success field will be set as false and message field will contain an error message.

\begin{lstlisting}[language=JavaScript]
{
	"seq"        : <number>,
	"type"       : "response",
	"request_seq": <number>,
	"command"    : <command>,
	"body"       : { ... },
	"running"    : <is the VM running after sending the message>,
	"success"    : <boolean indicating success>,
	"message"    : <error message>
}
\end{lstlisting}

An event packet has the following structure.

\begin{lstlisting}[language=JavaScript]
{
	"seq"  : <number>,
	"type" : "event",
	"event": <event name>,
	"body" : ...
}
\end{lstlisting}

\subsection {V8 Debugger Protocol Features}
V8 debugger has various commands and events providing detailed runtime information. However, this research focus on the data structure visualization. Only following features are used in order to implement basic debugger features.

\begin {itemize}
\item \textbf{Request} continue
\item \textbf{Request} evaluate
\item \textbf{Request} lookup
\item \textbf{Request} source
\item \textbf{Request} setbreakpoint
\item \textbf{Request} clearbreakpoint
\item \textbf{Event} break
\item \textbf{Event} exception
\end {itemize} 

Request "continue" makes V8 start running or stepping forward, including stepping in, stepping over, and stepping out. Although step count can be indicated in the arguments, we always set it as 1.

Request "evaluate" is used to evaluate a expression. However, if the result is object type that contains other fields, all fields will be represented as their object handle. Hence we have to use request "lookup" to lookup objects based on their handle. As a result, we can get the deep copy of any object by recursively using request "lookup".

Request "source" is used to retrieve source code for a frame. Frame and code range have to be indicated in the arguments. Note here that each script file running on node.js is wrapped within a wrapper function. Hence we have to remove the header and tail before showing it to users.

Request "setbreakpoint" is used to add breakpoint. Target file/function and line number are essential here. Request "clearbreakpoint" is used to remove breakpoint set by request "setbreakpoint". Breakpoint number which can be received from request "setbreakpoint" has to be indicated in the arguments. There also exists other kinds of requests like request "backtrace" which is used to require stacktrace information, request "frame" which is used to require frame information and so on.

\subsection {Response Object Serialization}
As discussed in 4.1.2, request "evaluate" and "lookup" may contain objects as part of the body. All objects are assigned with an ID called handle. Object identity\cite{Khoshafian:1986:OI:960112.28739} is that property of an object which distinguishes each object from all others. Although the handle can be used to identify objects here, it has a certain lifetime after which it will no longer refer to the same object. The lifetime of handles are recycled for each debug event.

For objects serialized they all contains two basic elements, handle and type. Each object has following the structure.

\begin{lstlisting}[language=JavaScript]
{
	"handle": <number>,
  "type"  : <"undefined", "null", "boolean", "number",
  					"string", "object", "function">,
  ...
}
\end{lstlisting}

For primitive JavaScript types, the value is part of the result.

\begin {itemize}
\item 0 \(\rightarrow\)
\begin{lstlisting}[language=JavaScript]
{
	"handle": <number>,
  "type"  : "number",
  "value" : 0
}
\end{lstlisting}
\item "hello" \(\rightarrow\)
\begin{lstlisting}[language=JavaScript]
{
	"handle": <number>,
  "type"  : "string",
  "value" : "hello"
}
\end{lstlisting}
\item true \(\rightarrow\)
\begin{lstlisting}[language=JavaScript]
{
	"handle": <number>,
  "type"  : "boolean",
  "value" : true
}
\end{lstlisting}
\item null \(\rightarrow\)
\begin{lstlisting}[language=JavaScript]
{
	"handle": <number>,
  "type"  : "null",
}
\end{lstlisting}
\item undefined \(\rightarrow\)
\begin{lstlisting}[language=JavaScript]
{
	"handle": <number>,
  "type"  : "undefined",
}
\end{lstlisting}
\end {itemize}

An object is encoded with additional information.

\{a:1,b:2\} \(\rightarrow\)
\begin{lstlisting}[language=JavaScript]
{
	"className"          : "Object"
	"constructorFunction": { "ref": <number> },
	"handle"             : <number>,
	"properties"         : [{ "ref": <number> }, ...],
	"protoObject"        : { "ref": <number> },
	"prototypeObject"    : { "ref": <number> },
	"text"               : "#<Object>",
	"type"               : "object"
}
\end{lstlisting}

An function is encoded as an object with additional information in the properties name, inferredName, source and script.

function()\{\} \(\rightarrow\)
\begin{lstlisting}[language=JavaScript]
{
	"handle"             : <number>,
	"type"               : "function",
	"className"          : "Function",
	"constructorFunction": { "ref": <number> },
	"protoObject"        : { "ref": <number> },
	"prototypeObject"    : { "ref": <number> },
	"name"               : "",
	"inferredName"       : "",
	"source"             : "function(){}",
	"script"             : { "ref": <number> },
	"scriptId"           : <number>,
	"position"           : <number>,
	"line"               : <number>,
	"column"             : <number>,
	"properties"         : [{
														"name": <string>,
														"ref" : <number>
													}, ...]
}
\end{lstlisting}

\section {System Architecture}
This is a full-stack JavaScript system which consists of three components.

\begin{enumerate}
	\item The debuggee node.js program is running on V8.
	\item Server is also running on node.js.
	\item Client is running on browser. 
\end{enumerate}

Server side is responsible for starting running debuggee program along with V8 debugger and communicate with it using V8 debugger protocol. It responds to the client and require according information from V8 debugger. When response arrives, this component is also responsible for informing client to update along with required information. Because the communication with V8 debugger is an asynchronized process whereas the system logic is basically synchronized, this component must be able to handle it carefully. Client side provides IDE-like debugging experience with an embedded editor to show the source code of the debuggee program, breakpoint management and series of stepping buttons. It also be responsible for visualization work. Figure \ref{fig: System Architecture} shows the system architecture.

\begin {figure} \centering
  \includegraphics [width=1.0\linewidth] {img/system}
  \caption {System architecture}
  \label {fig: System architecture}
\end {figure}

\subsection {Implementation Technique}
Protocol.js ensures data integrity in that some responses may be separated into several chunks. Client.js is provides basic features by encapsulating the communication with V8 debugger. Animator module generates and updates graph for target object with the given script. It is feasible to build different interfaces, like command interface or GUI interface.

On the client side, we used bootstrap for faster and easier web GUI development. Visualization related work is finished by D3.js which is an open-source project about visualizing. D3.js is a JavaScript library for manipulating documents based on data. D3 helps you bring data to life using HTML, SVG and CSS. D3’s emphasizes on combining powerful visualization components with a data-driven approach to DOM manipulation.

We also used other open-source libraries and tools like underscore, async.js, jquery, require.js, buckets, grunt, jasmine to help the development work on both sides.

\subsection {Asynchronized Communication Mechanism}
Another mentionable implementation technique used here is the asynchronized communication mechanism. Most I/O-related API containing TCP/IP communication provided by node.js is in asynchronized way. On the other hand, all requests sent from client via websocket is also in asynchronized way. As a result, it is very complicated to keep program running correctly because the overall control flow across three components is based on synchronization logic. Firstly, blocking message queue is used to handle the requests from client. All requests are stored in the queue and will be handled one by one. On the server side, Async.js is used to help manage asynchronized code. Although node.js is famous for its speed and single-thread model, its coding style is difficult to maintain on account of endless nested callback. Async.js is a tool to help alleviate this problem.

\section {VisualJS Parser}
We use PEG.js here to parse VisualJS code. PEG.js is a simple parser generator based on parsing expression grammars \cite{Ford:2004:PEG:982962.964011} for JavaScript that produces fast parsers with excellent error reporting. It is used to process complex data or computer languages and build transformers, interpreters, compilers and other tools.

\subsection {Using the Parser}
PEG.js generates parser from a grammar that describes expected input and can specify what the parser returns using semantic actions on matched parts of the input. Generated parser itself is a JavaScript object with a simple API.

Generated parser can be used by calling the parse method with an input string as a parameter. The method called parse will return a parse result or throw an exception if the target is invalid. The exception contains location information and other detailed error messages.

\subsection {Grammar Syntax and Semantics}
The grammar syntax is similar to JavaScript in that it is not line-oriented and ignores whitespace between tokens.

An example of parsing simple arithmetic expressions like 6/(1+2) is shown as follows. A parser generated from the grammar is able to calculate the expression results.

\begin{lstlisting}
start
  = additive

additive
  = left:multiplicative "+" right:additive {
  	return left + right;
  }
 	/ multiplicative

multiplicative
  = left:primary "*" right:multiplicative {
  	return left * right;
  }
  / primary

primary
  = "(" additive:additive ")" {
  	return additive;
  }
  / integer

integer
  = digits:[0-9]+ {
  	return parseInt(digits.join(""), 10);
  }
\end{lstlisting}

\subsection {Parsing Expression Types}
There are a series of parsing types, and some of them contain subexpressions forming a recursive structure:

\begin{itemize}
\item\textbf{"literal"}

Match literal string and return it.
\item\textbf{.}

Match an arbitrary character and return it as a string.
\item\textbf{[characters]}

Match one character from a set and return it as a string. We can set a range on characters. For example, [a-z] means all lowercase letters. Preceding the characters with $^\wedge$ inverts the matched set. For example, [$^\wedge$a-z] means all character but lowercase letters.
\item\textbf{( expression )}

Match a subexpression and return its match result.
\item\textbf{expression *}

Match zero or more repetitions of the expression and return the match results in an array. The matching is greedy. The parser will match the expression as many times as possible.
\item\textbf{expression +}

Match one or more repetitions of the expression and return the match results in an array. The matching is greedy. The parser will match the expression as many times as possible.
\item\textbf{expression ?}

Try to match the expression. If the match succeeds, the match result will be returned, otherwise null will be returned.
\item\textbf{$\&$ expression}

Try to match the expression. If the match succeeds, just return undefined and do not advance the parser position, otherwise consider the match failed.
\item\textbf{! expression}

Try to match the expression. If the match does not succeed, just return undefined and do not advance the parser position, otherwise consider the match failed.
\item\textbf{\$ expression}

Try to match the expression. If the match succeeds, return the matched string instead of the match result.
\item\textbf{label : expression}

Match the expression and remember its match result under given label.
\item\textbf{expression1 expression2 ... expressionn}

Match a sequence of expressions and return their match results in an array.
\item\textbf{expression \{ action \}}

Match the expression. If the match is successful, run the action, otherwise consider the match failed. The action is a piece of JavaScript code that is executed as if it was inside a function. It gets the match results of labeled expressions in preceding expression as its arguments. The action should return some JavaScript value using the return statement. This value is considered match result of the preceding expression.
\item\textbf{expression1 / expression2 / ... / expressionn}

Try to match the first expression, if it does not succeed, try the second one, etc. Return the match result of the first successfully matched expression. If no expression matches, consider the match failed.
\end{itemize}



\chapter {Examples}
To best explain how VisualJS converts an object to visual shapes, some typical examples will be talked below.

\section {Quick Sort}

\section {AVL Tree}

\section {Math Expression Tree}

\chapter {Summary}
\label {Summary}

We proposed that both command-line debuggers and modern IDEs are based on character-based expressions evaluation.

\section {Future Work}
First, the system handles only one data structure at the same time. However, data structures designed by real-world softwares are usually more complicated. Fore example, one container object is defined in list structure, and each element of the list is defined in tree structure, whose node is a user-defined data structure. These kinds of compound data structures can not be handled by current system.

Some common system enhancements could be some help. For example, the system supports only two data structures, tree and bar chart, on the current stage. More data structures need to be covered before the system could be used to debug practical programs.

Finally, system's effectiveness in terms of helping programmers understand programs, must be evaluated with practical users. Especially, whether and how much our system could simplify and accelerate the debugging process should be examined.
\chapter*{Acknowledgements}

Foremost, I would like to express my sincere gratitude to my advisor Prof. Ken Wakita for the continuous support of my Master study and research, for his patience, motivation, enthusiasm, and immense knowledge. I could not have imagined having a better advisor and mentor for my master study.

Besides my advisor, I would like to thank my fellow labmates for the stimulating discussions, and for all the fun we have had in the last two years. 

Last but not the least, I would like to thank my parents, for supporting me spiritually throughout my life.

\input {doc/appendix}

\bibliographystyle {plain}
\bibliography {/Users/junjianxu/Dropbox/projects/visualjs/documents/reference}

\end {document}
