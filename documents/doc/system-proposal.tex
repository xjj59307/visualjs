\chapter {System Proposal}

\section {Declarative language}

\section {Animation Mechanism}
To generate valid data structure animation, we must prove the consistency between the visualization and program. A program is composed of a series of instructions which are executed orderly. Under the control of debugger, program is being paused until step requests come. Every time the program pauses, it represents a new program state. What we need to prove is that every time the new program state has been generated, the animation will respond to it correctly.

Object graph details the relationships between objects. We can get one object graph \(G = (V, E)\) for a given object at certain program state. The vertex set $V_p$ is defined by objects which refer to that given object directly or indirectly, and the edge set $E$ is defined by reference relationships among those objects.

By iterating the object graph using next actions, we will get a subgraph of it, \(G_p = (V_p, E_p)\). Visual nodes are defined based on this graph. 

Visual nodes including two types, node and edge, are created by create actions. They are also a graph, \(G_v = (V_v, E_v)\). The vertex set is defined by visual nodes of node type. They are related to $V_p$. The relationships can be defined as a non-injective and surjective function \(f:V_p\rightarrow V_v\) (\textbf{\textit{TODO}}: illustration). The edge set is defined by visual nodes of edge type.

\section {Data Structure Visualization and Animation}
