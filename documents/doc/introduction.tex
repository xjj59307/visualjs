\chapter {Introduction}

\section {Motivation}
Program visualization systems translate program into visual shapes. It is often used in algorithm animation systems where algorithm behavior is visualized by producing the abstraction of the data and the operations of the algorithm. Such visualization also can be used in a debugger, as it allows for better understanding on the behavior of the program and help perceive which parts of the program does not function correctly.

A debugger is used to test and debug target programs. It offers many sophisticated functions such as running a program step by step, pausing the program at some event or specified instruction by means of a breakpoint, and examining the current state by evaluating the values of expressions or checking the stack trace. However, the character-based expressions evaluation provides limited insight into the current program state. Modern IDEs intend to solve this problem by offering a dedicated view to watch expressions. As a typical example of object-oriented programming language, when the variable being watched is of reference type, its fields is listed in a tree-view table. The root of this list represents the value itself. If a field is also of reference type, it can be further expanded in the same manner. Fields of primitive type can not be further expanded.

However, such visualization still has deficiencies. Firstly, all of the objects are visualized in a tree-view table, and hence lack of important semantic information. Secondly, The whole view will be refreshed while stepping through the program. It is nearly impossible to check the modification parts and find the relevance between two succeeding states.

This paper examines how a debugger can visualize data structure and generate smooth animation automatically while according object is modified via externally supplied semantic information. Instead of displaying object in a general way, we wish to produce more helpful and informative visualization.

\section {Main Contribution}

\section {Outline of the Thesis}