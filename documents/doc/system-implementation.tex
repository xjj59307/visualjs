\chapter {System Implementation}

\section {V8 Debugger Protocol}
V8 is able to debug the JavaScript code running in it. The debugger related API can be used in two ways, a function based API using JavaScript objects and a message based API using JSON based protocol. The function based API is for in-process usage, while the message based API is for out-process usage. This system is implemented with message based API. The protocol packet is defined in JSON format and can to be converted to string.

\subsection {Protocol Packet Format}
All packets have two basic elements called seq and type. The seq field holds the consecutive assigned sequence number of the packet. And type field is a string value representing the packet is request, response or event. Each request will receive a response with the same request seq number as long as the connection still works. And additional events will be generated on account of particular requests or system errors. Each packet has the following structure.

\begin{lstlisting}[language=JavaScript]
{
	"seq" : <number>,
	"type": <type>,
	...
}
\end{lstlisting}

A request packet has the following structure.

\begin{lstlisting}[language=JavaScript]
{
	"seq"      : <number>,
	"type"     : "request",
	"command"  : <command>,
	"arguments": { ... }
}
\end{lstlisting}

A response packet has the following structure. If command fails, the success field will be set as false and message field will contain an error message.

\begin{lstlisting}[language=JavaScript]
{
	"seq"        : <number>,
	"type"       : "response",
	"request_seq": <number>,
	"command"    : <command>,
	"body"       : { ... },
	"running"    : <is the VM running after sending the message>,
	"success"    : <boolean indicating success>,
	"message"    : <error message>
}
\end{lstlisting}

An event packet has the following structure.

\begin{lstlisting}[language=JavaScript]
{
	"seq"  : <number>,
	"type" : "event",
	"event": <event name>,
	"body" : ...
}
\end{lstlisting}

\subsection {V8 Debugger Protocol Features}
V8 debugger has various commands and events providing detailed runtime information. However, this research focus on the data structure visualization. Only following features are used in order to implement basic debugger features.

\begin {itemize}
\item \textbf{Request} continue
\item \textbf{Request} evaluate
\item \textbf{Request} lookup
\item \textbf{Request} source
\item \textbf{Request} setbreakpoint
\item \textbf{Request} clearbreakpoint
\item \textbf{Event} break
\item \textbf{Event} exception
\end {itemize} 

Request "continue" makes V8 start running or stepping forward, including stepping in, stepping over, and stepping out. Although step count can be indicated in the arguments, we always set it as 1.

Request "evaluate" is used to evaluate a expression. However, if the result is object type that contains other fields, all fields will be represented as their object handle. Hence we have to use request "lookup" to lookup objects based on their handle. As a result, we can get the deep copy of any object by recursively using request "lookup".

Request "source" is used to retrieve source code for a frame. Frame and code range have to be indicated in the arguments. Note here that each script file running on node.js is wrapped within a wrapper function. Hence we have to remove the header and tail before showing it to users.

Request "setbreakpoint" is used to add breakpoint. Target file/function and line number are essential here. Request "clearbreakpoint" is used to remove breakpoint set by request "setbreakpoint". Breakpoint number which can be received from request "setbreakpoint" has to be indicated in the arguments. There also exists other kinds of requests like request "backtrace" which is used to require stacktrace information, request "frame" which is used to require frame information and so on.

\subsection {Response Object Serialization}
As discussed in 4.1.2, request "evaluate" and "lookup" may contain objects as part of the body. All objects are assigned with an ID called handle. Object identity\cite{Khoshafian:1986:OI:960112.28739} is that property of an object which distinguishes each object from all others. Although the handle can be used to identify objects here, it has a certain lifetime after which it will no longer refer to the same object. The lifetime of handles are recycled for each debug event.

For objects serialized they all contains two basic elements, handle and type. Each object has following the structure.

\begin{lstlisting}[language=JavaScript]
{
	"handle": <number>,
  "type"  : <"undefined", "null", "boolean", "number",
  					"string", "object", "function">,
  ...
}
\end{lstlisting}

For primitive JavaScript types, the value is part of the result.

\begin {itemize}
\item 0 \(\rightarrow\)
\begin{lstlisting}[language=JavaScript]
{
	"handle": <number>,
  "type"  : "number",
  "value" : 0
}
\end{lstlisting}
\item "hello" \(\rightarrow\)
\begin{lstlisting}[language=JavaScript]
{
	"handle": <number>,
  "type"  : "string",
  "value" : "hello"
}
\end{lstlisting}
\item true \(\rightarrow\)
\begin{lstlisting}[language=JavaScript]
{
	"handle": <number>,
  "type"  : "boolean",
  "value" : true
}
\end{lstlisting}
\item null \(\rightarrow\)
\begin{lstlisting}[language=JavaScript]
{
	"handle": <number>,
  "type"  : "null",
}
\end{lstlisting}
\item undefined \(\rightarrow\)
\begin{lstlisting}[language=JavaScript]
{
	"handle": <number>,
  "type"  : "undefined",
}
\end{lstlisting}
\end {itemize}

An object is encoded with additional information.

\{a:1,b:2\} \(\rightarrow\)
\begin{lstlisting}[language=JavaScript]
{
	"className"          : "Object"
	"constructorFunction": { "ref": <number> },
	"handle"             : <number>,
	"properties"         : [{ "ref": <number> }, ...],
	"protoObject"        : { "ref": <number> },
	"prototypeObject"    : { "ref": <number> },
	"text"               : "#<Object>",
	"type"               : "object"
}
\end{lstlisting}

An function is encoded as an object with additional information in the properties name, inferredName, source and script.

function()\{\} \(\rightarrow\)
\begin{lstlisting}[language=JavaScript]
{
	"handle"             : <number>,
	"type"               : "function",
	"className"          : "Function",
	"constructorFunction": { "ref": <number> },
	"protoObject"        : { "ref": <number> },
	"prototypeObject"    : { "ref": <number> },
	"name"               : "",
	"inferredName"       : "",
	"source"             : "function(){}",
	"script"             : { "ref": <number> },
	"scriptId"           : <number>,
	"position"           : <number>,
	"line"               : <number>,
	"column"             : <number>,
	"properties"         : [{
														"name": <string>,
														"ref" : <number>
													}, ...]
}
\end{lstlisting}

\section {System Architecture}
This is a full-stack JavaScript system which consists of three components.

\begin{enumerate}
	\item The debuggee node.js program is running on V8.
	\item Server is also running on node.js.
	\item Client is running on browser. 
\end{enumerate}

Server side is responsible for starting running debuggee program along with V8 debugger and communicate with it using V8 debugger protocol. It responds to the client and require according information from V8 debugger. When response arrives, this component is also responsible for informing client to update along with required information. Because the communication with V8 debugger is an asynchronized process whereas the system logic is basically synchronized, this component must be able to handle it carefully. Client side provides IDE-like debugging experience with an embedded editor to show the source code of the debuggee program, breakpoint management and series of stepping buttons. It also be responsible for visualization work. Figure \ref{fig: System Architecture} shows the system architecture.

\begin {figure} \centering
  \includegraphics [width=1.0\linewidth] {img/system}
  \caption {System architecture}
  \label {fig: System architecture}
\end {figure}

\subsection {Implementation Technique}
Protocol.js ensures data integrity in that some responses may be separated into several chunks. Client.js is provides basic features by encapsulating the communication with V8 debugger. Animator module generates and updates graph for target object with the given script. It is feasible to build different interfaces, like command interface or GUI interface.

On the client side, we used bootstrap for faster and easier web GUI development. Visualization related work is finished by D3.js which is an open-source project about visualizing. D3.js is a JavaScript library for manipulating documents based on data. D3 helps you bring data to life using HTML, SVG and CSS. D3’s emphasizes on combining powerful visualization components with a data-driven approach to DOM manipulation.

We also used other open-source libraries and tools like underscore, async.js, jquery, require.js, buckets, grunt, jasmine to help the development work on both sides.

\subsection {Asynchronized Communication Mechanism}
Another mentionable implementation technique used here is the asynchronized communication mechanism. Most I/O-related API containing TCP/IP communication provided by node.js is in asynchronized way. On the other hand, all requests sent from client via websocket is also in asynchronized way. As a result, it is very complicated to keep program running correctly because the overall control flow across three components is based on synchronization logic. Firstly, blocking message queue is used to handle the requests from client. All requests are stored in the queue and will be handled one by one. On the server side, Async.js is used to help manage asynchronized code. Although node.js is famous for its speed and single-thread model, its coding style is difficult to maintain on account of endless nested callback. Async.js is a tool to help alleviate this problem.

\section {VisualJS Parser}
We use PEG.js here to parse VisualJS code. PEG.js is a simple parser generator based on parsing expression grammars \cite{Ford:2004:PEG:982962.964011} for JavaScript that produces fast parsers with excellent error reporting. It is used to process complex data or computer languages and build transformers, interpreters, compilers and other tools.

\subsection {Using the Parser}
PEG.js generates parser from a grammar that describes expected input and can specify what the parser returns using semantic actions on matched parts of the input. Generated parser itself is a JavaScript object with a simple API.

Generated parser can be used by calling the parse method with an input string as a parameter. The method called parse will return a parse result or throw an exception if the target is invalid. The exception contains location information and other detailed error messages.

\subsection {Grammar Syntax and Semantics}
The grammar syntax is similar to JavaScript in that it is not line-oriented and ignores whitespace between tokens.

An example of parsing simple arithmetic expressions like 6/(1+2) is shown as follows. A parser generated from the grammar is able to calculate the expression results.

\begin{lstlisting}
start
  = additive

additive
  = left:multiplicative "+" right:additive {
  	return left + right;
  }
 	/ multiplicative

multiplicative
  = left:primary "*" right:multiplicative {
  	return left * right;
  }
  / primary

primary
  = "(" additive:additive ")" {
  	return additive;
  }
  / integer

integer
  = digits:[0-9]+ {
  	return parseInt(digits.join(""), 10);
  }
\end{lstlisting}

\subsection {Parsing Expression Types}
There are a series of parsing types, and some of them contain subexpressions forming a recursive structure:

\begin{itemize}
\item\textbf{"literal"}

Match literal string and return it.
\item\textbf{.}

Match an arbitrary character and return it as a string.
\item\textbf{[characters]}

Match one character from a set and return it as a string. We can set a range on characters. For example, [a-z] means all lowercase letters. Preceding the characters with $^\wedge$ inverts the matched set. For example, [$^\wedge$a-z] means all character but lowercase letters.
\item\textbf{( expression )}

Match a subexpression and return its match result.
\item\textbf{expression *}

Match zero or more repetitions of the expression and return the match results in an array. The matching is greedy. The parser will match the expression as many times as possible.
\item\textbf{expression +}

Match one or more repetitions of the expression and return the match results in an array. The matching is greedy. The parser will match the expression as many times as possible.
\item\textbf{expression ?}

Try to match the expression. If the match succeeds, the match result will be returned, otherwise null will be returned.
\item\textbf{$\&$ expression}

Try to match the expression. If the match succeeds, just return undefined and do not advance the parser position, otherwise consider the match failed.
\item\textbf{! expression}

Try to match the expression. If the match does not succeed, just return undefined and do not advance the parser position, otherwise consider the match failed.
\item\textbf{\$ expression}

Try to match the expression. If the match succeeds, return the matched string instead of the match result.
\item\textbf{label : expression}

Match the expression and remember its match result under given label.
\item\textbf{expression1 expression2 ... expressionn}

Match a sequence of expressions and return their match results in an array.
\item\textbf{expression \{ action \}}

Match the expression. If the match is successful, run the action, otherwise consider the match failed. The action is a piece of JavaScript code that is executed as if it was inside a function. It gets the match results of labeled expressions in preceding expression as its arguments. The action should return some JavaScript value using the return statement. This value is considered match result of the preceding expression.
\item\textbf{expression1 / expression2 / ... / expressionn}

Try to match the first expression, if it does not succeed, try the second one, etc. Return the match result of the first successfully matched expression. If no expression matches, consider the match failed.
\end{itemize}


